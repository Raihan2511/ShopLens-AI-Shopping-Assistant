\section{Dataset Description}
\begin{frame}{}
  \Huge
  \centering
  \textbf{Dataset Description}
  \normalsize
\end{frame}

\begin{frame}{Source - Fashionpedia Dataset}

  \begin{figure}[H]
    \centering
    \includegraphics[width=0.5\textwidth]{images/hugging-face.png}
  \end{figure}
  Link - \footnotesize \href{https://huggingface.co/datasets/detection-datasets/fashionpedia}{https://huggingface.co/datasets/detection-datasets/fashionpedia}\\
  \normalsize

  This dataset was constructed by fashion experts, containing \textbf{46,781 images} with \textbf{342,182 bounding boxes}. It is used for fine-tuning the YOLOS model for object detection tasks within the fashion domain.

\end{frame}

\begin{frame}{Dataset Statistics}

\begin{columns}[c] % 'c' for vertical centering
    \column{0.4\textwidth}
    \centering
    \textbf{General Information}

    \vspace{0.3em}
    \begin{tabular}{|l|r|}
    \hline
    \textbf{Total Images} & 46,781 \\
    \hline
    \textbf{Total BBoxes} & 342,182 \\
    \hline
    \end{tabular}

    \column{0.5\textwidth}
    \centering
    \textbf{Data Split}

    \vspace{0.3em}
    \begin{tabular}{|l|r|}
    \hline
    \textbf{Training Set} & 45,623 images \\
    \hline
    \textbf{Validation Set} & 1,158 images \\
    \hline
    \end{tabular}
\end{columns}


\begin{table}[h!]
  \centering
  \textbf{Each sample contains:}
\begin{tabular}{|l|l|}
\hline
\textbf{Field} & \textbf{Description} \\
\hline
\textbf{image\_id} & Unique image identifier \\
\hline
\textbf{image} & RGB image \\
\hline
\textbf{width, height} & Image dimensions \\
\hline
\textbf{objects} & Detected objects metadata \\
\hline
\end{tabular}
\end{table}

\textbf{Object Metadata:}
\begin{itemize}
    \item \textbf{bbox\_id}, \textbf{category}, \textbf{bbox} (Pascal VOC format), \textbf{area}
\end{itemize}

\end{frame}

\begin{frame}{Dataset Statistics}
  \textbf{46 Fashion Categories}

  \scriptsize
  \begin{multicols}{4}
  \begin{itemize}
      \item shirt, blouse
      \item top, t-shirt, sweatshirt
      \item sweater
      \item cardigan
      \item jacket
      \item vest
      \item pants
      \item shorts
      \item skirt
      \item coat
      \item dress
      \item jumpsuit
      \item cape
      \item glasses
      \item hat
      \item headband, hair accessory
      \item tie
      \item glove
      \item watch
      \item belt
      \item leg warmer
      \item tights, stockings
      \item sock
      \item shoe
      \item bag, wallet
      \item scarf
      \item umbrella
      \item hood
      \item collar
      \item lapel
      \item epaulette
      \item sleeve
      \item pocket
      \item neckline
      \item buckle
      \item zipper
      \item applique
      \item bead
      \item bow
      \item flower
      \item fringe
      \item ribbon
      \item rivet
      \item ruffle
      \item sequin
      \item tassel
  \end{itemize}
  \end{multicols}
\end{frame}


\begin{frame}{Source - Fashion Product Images Dataset}
  \begin{figure}[H]
    \centering
    \includegraphics[width=0.3\textwidth]{images/kaggle.png}
  \end{figure}
  Link - \footnotesize \href{https://www.kaggle.com/datasets/paramaggarwal/fashion-product-images-dataset}{https://www.kaggle.com/datasets/paramaggarwal/fashion-product-images-dataset}\\
  \normalsize

  It contains a total of \textbf{1,05,542 fashion products}, almost each with a image and associated metadata.
\end{frame}

\begin{frame}{Dataset Statistics}
  \small

    \centering
    \renewcommand{\arraystretch}{1.2}
    \begin{tabular}{|p{0.4\textwidth}|p{0.5\textwidth}|}
    \hline
    \textbf{Field Name} & \textbf{Description} \\
    \hline
    \textbf{article\_id} & Unique product identifier \\
    \hline
    \textbf{prod\_name} & Product name \\
    \hline
    \textbf{product\_type\_name} & Broad category of the product \\
    \hline
    \textbf{colour\_group\_name} & Primary color grouping \\
    \hline
    \textbf{department\_name} & Department like Menswear, Ladieswear \\
    \hline
    \textbf{section\_name} & Section within department \\
    \hline
    \textbf{garment\_group\_name} & Garment group classification \\
    \hline
    \textbf{detail\_desc} & Detailed product description \\
    \hline
    \end{tabular}

\end{frame}